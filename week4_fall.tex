\documentclass[]{article}
\usepackage{graphicx}

\parindent=0pt
\usepackage[margin=0.5in]{geometry}

\begin{document}
\pagestyle{empty}
{\large\textbf{Research Notes}}
\begin{itemize}
    \item[*] Created on Sept 21, 2014
    \item[*] Modified on \today
    \item[*] Author info: Boyou Zhou\\
             8 St Mary's St, PHO 340, Boston, MA 02215\\
             Email: bobzhou@bu.edu, Phone: 617-678-8480
\end{itemize}


\rule[-0.1cm]{7.5in}{0.01cm}\\
\\
\noindent \textbf{Sept 3,2014}
\textit{An Efficient Real Time Fault Detection and Tolerance Framework Validated on the Intel SCC Processor}
\indent		\begin{itemize}
            \item \textit{main motivation} Current fault detection techniques are not based on the time on the fly.
            In terms of computational resources using, it is more efficient to do it on the run. 

            \item \textit{main idea} This paper emphasizes on the real time process networks. The author proposes the
            idea to create a buffer and by measuring the size of the buffer to have the fault detection.
            Error detection is based on building up a replica of the existing network.
            When the error has been detected, the buffer will be full and blocking the incoming process, and as a result
            the selector has been flagged as blocker.

            \item \textit{Kahn Process Network} the process network is independent of the timing of the network

            \item \textit{critics} The proposed idea has paid to much price for the error detection. The entire replica
            of a process flow costs too muck.

        \end{itemize}
\end{document}

