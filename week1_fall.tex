\documentclass[]{article}
\usepackage{graphicx}

\parindent=0pt
\usepackage[margin=0.5in]{geometry}

\begin{document}
\pagestyle{empty}
{\large\textbf{Research Notes}}
\begin{itemize}
    \item[*] Created on Sept 3, 2014
    \item[*] Modified on \today
    \item[*] Author info: Boyou Zhou\\
             8 St Mary's St, PHO 340, Boston, MA 02215\\
             Email: bobzhou@bu.edu, Phone: 617-678-8480
\end{itemize}


\rule[-0.1cm]{7.5in}{0.01cm}\\
\\
\noindent \textbf{Sept 3,2004}
\textit{A 10.4pJ/b (32,8) LDPC Decoder with Time-Domain Analog and Digital Mixed-Signal Processing}
\indent		\begin{itemize}
			\item \textit{Main motivation}  Most of the analog computation circuits have a much higher efficiency in computational operation.
            The paper argued that the time-domain analog and digital mixed signal will be more effective in the computation. 
            Thus the paper has used a LDPC decoder as the example.

			\item \textit{LDPC decoder} LDPC is a very efficient coding method that has been applied in Wifi standard.
            The basic idea is to multiply the info that needs to be coded with a polarity check matrix to form the coded info.
            The circuit is very easy to design, as it just consists of an adder and multiplication and a feedback loop.
            The decoding part is using possibility decoding, so it will involve a little programming.
            I think LDPC decoder can be a small project with the matlab decoding.

            \item \textit{Time-to-digital} Use a counter to count how many clock cycles have passed to measure the time that has passed.
            And this value is converted to a digital representation. 
            TDC requires highly accurate clock in order to achieve correct time to digital conversion.
            The fine measurements have not been read in this review. \textbf{Sept 3, 2004}

            \item \textit{Basic idea} The info that needs to be coded will be go into a DTC and converted to time info. 
            It will than going through a special designed LDPC decoder in which basic operations have been done in the analog domain.
            And then convert the info to the digital domain. 
            This wil reduce energy consumption, because all info have been done in the analog domain.

            \item \textit{Critics} The testbench is only for the LDPC decoder, for some other circuit it might not need such amount of clock signal transition.
            From a personal point of view, it is strongly recommanded for other circuit for testing. 
            It is a perfect idea for applying analog computional circuit to digital domain.
            But overhead might be a problem in other circuits, especially for the DTC and TDC.

        \end{itemize}

\end{document}
