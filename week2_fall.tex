\documentclass[]{article}
\usepackage{graphicx}

\parindent=0pt
\usepackage[margin=0.5in]{geometry}

\begin{document}
\pagestyle{empty}
{\large\textbf{Research Notes}}
\begin{itemize}
    \item[*] Created on Sept 6, 2014
    \item[*] Modified on \today
    \item[*] Author info: Boyou Zhou\\
             8 St Mary's St, PHO 340, Boston, MA 02215\\
             Email: bobzhou@bu.edu, Phone: 617-678-8480
\end{itemize}


\rule[-0.1cm]{7.5in}{0.01cm}\\
\\
\noindent \textbf{Sept 6,2014}
\textit{Wide I/O with 4096b TSVs Through an active silicon interposer with in-place waveform capturing}
\indent		\begin{itemize}
            \item \textit{main motivation} The reason why 3D stacking is not applicable towards the massive production is that TSV is to expensive for massive production.
            The interposer is active, it can be controlled actively through the control. 
            The main point is to use the interposer to detect the waveform throught the TSV in order to build up a test agent to test the communication between different layers.
            \item \textit{main idea} The idea is that the chip needs massive amount of veritical connections.
            The silicon interposers in the stack provide pitch in horizontal and veritical channels for the accomodating TSVs and uBunmps.
            In the paper, the author states well about how the chip is going to test and describe the testing agent function's behavior.
            \item \textit{critcs} The BIST is through the traditional design and most of the ideas are to modify the current idea into an interposer.
            Personally, I think it is a good paper to present the idea to work in the future, but the application for the 3D is very limited.
        \end{itemize}

\noindent \textbf{Sept 7,2014}
\textit{An 8MHz 75uA/MHz Zero leakage non-volatile logic based cortex mo MCU Soc exhibiting 100percent digital state retention at vdd=0}
\indent		\begin{itemize}
            \item \textit{main motivation} 
                TI proposed an idea to use the logic based memory which is also non-volatile.
                The system also reduces the wake up time when the vdd starts to power up from 0 voltage.
               
            \item \textit{main idea} 
                The idea is to use capacitor to store the information.
                The best part of the paper is that it use the control signal to control the writing operation towards the NV cells.
                The effective driver will accelerate the read and write time.
                
            \item \textit{Critics}
                The price pay to the area is enormouse. 
                The entire on cell has been increased into several capacitors.
                I did not lookinto this paper, but I think it is a good paper to start to understand fast read and write to NV cells. 
        \end{itemize}

\noindent \textbf{Sept 8,2014}
\textit{A High Performance eleiptic curve cryptographic processor over GF(p) with SPA Resistance}
\indent		\begin{itemize}

            \item \textit{SPA} Simple Power Analysis. It is related to the security problem of the hardware.\textbf{SPA attacks}

            \item \textit{Montgomery modular multiplication} It is an algorithm that proves to be much more efficient after several hundreds of bits.\textbf{interesting algorithm}
            
            \item \textit{eleiptic curve crytographic system} It is another cryptographic systems as an alternative towards RSA system. It can achieve 160 bits security as RSA can do
                in 1024 bit. Personally, I support to use the state of art cryptology for the hardware securty testbench. Once again, the cryptology involves the finite field study.
                Thus, I think, it is very important to read through the finite field study.

            \item \textit{main motivation} SPA has presented a great threat for hardware devices of public cryptosystems. By analyzing the power consumption, it is easy to detect
                the private key's value. That is why the paper proposes a better way to counter SPA. The Montgomery modular multiplication is used for high speed modular computation, which
                is needed in encryptography.

            \item \textit{Basic idea} The paper is about use the ECCS to build a system that can counter SPA. The idea is to build it.

            \item \textit{Critics} I do not know much about the security part. But I am sure there should be a way to breach it. Anyway, it is good way to start looking into the security
                problems. Understanding the public keys and private keys.

            \end{itemize}

\end{document}

