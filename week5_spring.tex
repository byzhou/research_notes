\documentclass[]{article}
\usepackage{graphicx}

\parindent=0pt
\usepackage[margin=0.5in]{geometry}

\begin{document}
\pagestyle{empty}
{\large\textbf{Research Notes}}
\begin{itemize}
    \item[*] Created on Feb 26, 2014
    \item[*] Modified on \today
    \item[*] Author info: Boyou Zhou\\
             8 St Mary's St, PHO 340, Boston, MA 02215\\
             Email: bobzhou@bu.edu, Phone: 617-678-8480
\end{itemize}


\rule[-0.1cm]{7.5in}{0.01cm}\\
\\
\noindent \textbf{Feb 26,2014}
\textit{a watermarking system for ip protection by buffer insertion technique}
\indent		\begin{itemize}
            \item \textit{main motivation} IP protection, of course.
            \item \textit{main idea} Using buffers for ip privacy. The number of
            buffers is the specifc due to design.
        \end{itemize}


\noindent \textbf{Feb 28,2014}
\textit{Provably Secure Active IC Metering Techniques for Piracy Avoidance and
Digital Rights Management}
\indent		\begin{itemize}
                \item \textit{IP core protection} 
                    \begin{itemize}
                        \item the functional description of the design
                        \item unique and unclonable IC identifiers
                    \end{itemize}
                \item \textit{enabling or disabling remotely} The locks are embedded
                in hardware int he form of FSM.
                \item demos on H.264 MPEG
                \item PUF design
                    \begin{itemize}
                        \item [18] silicon physical random function
                        \item [19] physical unclonable functions for device
                        authentication and secret key generation
                        \item [20] security based on physical unclonability and
                        disorder
                    \end{itemize}
                \item \textit{weak PUF} only generate a limited number of
                independent outputs.
                \item \textit{strong PUF} has many possible challenges and
                reponses.
        \end{itemize}

\noindent \textbf{Feb 28,2014}
\textit{Silicon Physical Random Functions}
\indent		\begin{itemize}
                \item \textit{PUF def} It is designed for the IC counterfeiting.
                The problem is very hard to characterize.
                \item \textit{Attacks}
                    \begin{itemize}
                        \item [*] Duplicate a PUF
                        \item [*] Direct measurement requires opening the
                        package and several layers of metals
                        \item [*] Measure a polynomial number of
                        adaptively-chosen challenges.
                        \item [*] Detection of the control system to break the
                        lock. The author said they have done some further
                        studies on that. see the paper being cited.
                    \end{itemize}
        \end{itemize}

\noindent \textbf{Feb 28,2014}
\textit{Silicon Physical Random Functions}
\indent		\begin{itemize}
            \item [*] \textit{avoid use of cells} cd
            /libraries/NangateOpenCellLibrary/libcells/AND2\_X4

                                            set\_attribute avoid true
        \end{itemize}
\end{document}

