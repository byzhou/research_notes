%   * -.-.-.-.-.-.-.-.-.-.-.-.-.-.-.-.-.-.-.-.
%    
%       *   File Name : week7_fall.tex
%       *   Purpose :
%       *   Creation Date : 10-13-2014
%       *   Last Modified : Thu 23 Oct 2014 11:03:36 AM EDT
%       *   Created By : Boyou Zhou
%    
%    The MIT License (MIT)
%    
%    Copyright (c) 2014 Boyou Zhou, John Harvard
%    
%    Permission is hereby granted, free of charge, to any person obtaining a copy
%    of this software and associated documentation files (the "Software"), to deal
%    in the Software without restriction, including without limitation the rights
%    to use, copy, modify, merge, publish, distribute, sublicense, and/or sell
%    copies of the Software, and to permit persons to whom the Software is
%    furnished to do so, subject to the following conditions:
%    
%    The above copyright notice and this permission notice shall be included in all
%    copies or substantial portions of the Software.
%    
%    THE SOFTWARE IS PROVIDED "AS IS", WITHOUT WARRANTY OF ANY KIND, EXPRESS OR
%    IMPLIED, INCLUDING BUT NOT LIMITED TO THE WARRANTIES OF MERCHANTABILITY,
%    FITNESS FOR A PARTICULAR PURPOSE AND NONINFRINGEMENT. IN NO EVENT SHALL THE
%    AUTHORS OR COPYRIGHT HOLDERS BE LIABLE FOR ANY CLAIM, DAMAGES OR OTHER
%    LIABILITY, WHETHER IN AN ACTION OF CONTRACT, TORT OR OTHERWISE, ARISING FROM,
%    OUT OF OR IN CONNECTION WITH THE SOFTWARE OR THE USE OR OTHER DEALINGS IN THE
%    SOFTWARE.
%    
%_._._._._._._._._._._._._._._._._._._._._.*/

\documentclass[]{article}
\usepackage{graphicx}

\parindent=0pt
\usepackage[margin=0.5in]{geometry}

\begin{document}
\pagestyle{empty}
{\large\textbf{Research Notes}}
\begin{itemize}
    \item[*] Created on Oct 20, 2014
    \item[*] Modified on \today
    \item[*] Author info: Boyou Zhou\\
             8 St Mary's St, PHO 340, Boston, MA 02215\\
             Email: bobzhou@bu.edu, Phone: 617-678-8480
\end{itemize}


\rule[-0.1cm]{7.5in}{0.01cm}\\
\\
\noindent \textbf{Oct 20,2014}
\textit{Hardware Trojan Detection through Golden Chip-Free Statistical Side-Channel Fingerprinting}
\indent		\begin{itemize}
                \item \textit{Utilize the Process Control Monitors} The monitors exists on the area
                        in-between die or on the die for process characterization.
                \item \textit{Non-linear Regression Models} The way to learn side-channel fingerprints
                        using Monte-Carlo simulation.
                \item \textit{Tail Modeling Methods} Enhance classification boundary.
                \item \textit{Critics} Use math models to enhance detection rate but without any 
                        real chips.
                \item \textit{main idea} Establish a systematic way to build up a multi-dimension
                        space area, where it is very easy to define a hyper-space region, which you
                        can tell which point is trojan free. All the axes are the side-channel factors.
                \item \textbf{our work concern} The HTs are added by reverse engineering. The problem 
                        is that if we add antennas, they will be discovered in the reverse engineered
                        part. And they are easily modified by adding those antenna cells.

            \end{itemize}

\noindent \textbf{Oct 20,2014}
\textit{something about running cadence}
\indent		\begin{itemize}
                \item \textit{run in command line is the best} Not only the gui will bring you a lot
                        of trouble but also it can cause failure.
                \item \textit{export gds in encounter} streamout yourFileName.gds
                \item \textit{import gds file in virtuoso} strmin -library AES\_trojan -strmFile /your
                        /path/of/the/gds/file/file.gds
                \item \textit{add filler} addFiller -cell FILLCELL\_X1 FILLCELL\_X2 FILLCELL\_X4 FILLCELL\_X8 FILLCELL\_X16 FILLCELL\_X32 -prefix FILL
                \item \textit{delete filler} deleteFiller -cell FILLCELL\_X1 FILLCELL\_X2 FILLCELL\_X4 FILLCELL\_X8 FILLCELL\_X16 FILLCELL\_X32 -prefix FILL
                \item \textit{select a group of things} selectObjByProp Module <Name>Match<Trojan>
                \item \textit{deleteModule} deleteModule Trojan
            \end{itemize}

\noindent \textbf{Oct 21,2014}
\textit{A novel RF fingerprinting approach for hardware integrated security}
\indent		\begin{itemize}
                \item \textit{main idea} The idea is to use RF fingerprint as a PUF. The RF response is 
                        unique with their signature. The extraction of the fingerprint is based on the 
                        scattered near-fields S-parameters.
                \item \textit{SIM} Subscriber Identity Module.
                \item \textit{PUF} The real PUF is in the purely-randomized distribution of particles
                        within the poly-styrene filling.
                \item \textit{method} They sweep over 20,000 points chosen for measurements, using 
                        $frac{n(n-1)}{2}$ combinations.
       		\end{itemize}

\noindent \textbf{Oct 23,2014}
\textit{PSR Enhancement Through Super Gain Boosting and Differential Feed-Forward Noise Cancellation
        a 65-nm CMOS LDO Regulator}
\indent		\begin{itemize}
                \item \textit{PSF} Power supply regection, is a term that is used to describe the fluctuations
                        in the input of the power supply's affection towards the outputs. 
                \item \textit{SGA} Super Gain Amplifier.
                \item \textit{ADC} ADC is a very noise-sensetive circuit.
                \item \textit{LDO small gain} The small loop gain of the LDO regulator produces a large steady-
                        state error at dc in its output.
                \item \textit{why analog circuit can not scale} Due to the short gate channel length, the output
                        resistance is too small and corresponding $V_{E}$ is also small.
                \item \textit{several ways to improve the gain with limited length} 
                        \begin{itemize}
                            \item \textit{multi-stages} This is limited by compensation problem.\textbf{I don't 
                                understand what compensation problem is. }
                            \item \textit{positive feedback}
                        \end{itemize}
                \item \textit{SGA} SGA is pretty interesting, worth designing sometimes.
                \item \textit{PFCM} Positive Feedback Current Mirror \textbf{interesting part}
            \end{itemize}
\end{document}

