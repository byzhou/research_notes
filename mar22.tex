\documentclass[]{article}
\usepackage{graphicx}
\usepackage{color}

\parindent=0pt
\usepackage[margin=0.5in]{geometry}

\begin{document}
\pagestyle{empty}
{\large\textbf{Research Notes}}
\begin{itemize}
    \item[*] Created on Mar 22, 2015
    \item[*] Modified on \today
    \item[*] Author info: Boyou Zhou\\
             8 St Mary's St, PHO 340, Boston, MA 02215\\
             Email: bobzhou@bu.edu, Phone: 617-678-8480
\end{itemize}


\rule[-0.1cm]{7.5in}{0.01cm}\\
\\

\noindent \textbf{Mar 22, 2015}
\textit{questions for ICCAD paper}
\indent		\begin{itemize}

                \item [*] Is there any papers have Hardware security problems
                published in ICCAD? How much percentages are the hardware
                security components compared to the CAD parts?
                \item Yes, there are papers. They have significant portion
                of hardware security parts.
                \item [*] What is the cost-effective solution?
                \item Memristor may not be the idea solution. It should be low
                cost thermal sensor accompanied by one time programming methods.
                The problem should be comparing the existing
                one-time-programming methods to new methods.
                \item [*] What is the state-of-art reverse engineering methods?
                \item It is possible to reverse engineering the entire ASIC.
                \item [*] Is it possible to reverse engneering the entire
                netlist?
                \item [*] What is the on-chip laser price?
                \item It is not cost-effective.
                \item {\color{red}The story goes with the TRNG. We use the metastability
                to generate the bit. The bit is written by state-of-art OTP (One
                Time Programmable Units), instead of memristor. The innovative
                part of the idea is that we use the laser seebeck voltage to
                generate stability from metastability.}
                \item programmable read-only memory (PROM) or field programmable
                read-only memory (FPROM) or one-time programmable non-volatile
                memory (OTP NVM) 
                \item modify the power analysis tool flow
                \item spectre command line simulation, as well as calibre
                
        \end{itemize}

\noindent \textbf{Mar 22, 2015}
\textit{Passive switching of eletromagnetic devices with memristors}
\indent		\begin{itemize}
        \item \textit{property} 
            \begin{itemize}
                \item The memristor directly links the charge and the flux. Thus it can
                converts flux information into charge information.             
                \item The $\beta$ constant is related to length of the memristor squared
                divided by the average ion mobility.
                \item The $\beta$ constant varies linearly with charge between the
                dopant barrier position and is bonded by the physical dimensions.
                \begin{equation}
                    R[x(t)] = x(t)R_{on}+[1-x(t)]R_{off}
                \end{equation}
                \item $R[x(t)]\in[0,1]$ defines the change from high res to low
                res.
            \end{itemize}
        \item The author is designing a rf signal applied to write the
        memristor.
        \item The rf signal applied to the memristor is pretty strong. It can
        not be written easily.
        \item It is hoped to be used as frequency selective surface.

        \end{itemize}

\noindent \textbf{Mar 22, 2015}
\textit{Temperature Tracking:An Innovative Run-Time Approach for Hardware Trojan
    Detection}
\indent		\begin{itemize}
                \item \textit{Main purpose of reading this paper} Try to decide
                how much percentages do the CAD components take compared to HT
                parts.
            \end{itemize}

\noindent \textbf{Mar 22, 2015}
\textit{The State-of-Art in IC Reverse Engineering}
\indent     \begin{itemize}
                \item types of RE
                \begin{itemize}
                    \item \textit{device depot} Dissolving packages
                    \item \textit{delayering IC}
                    \item \textit{imaging}
                    \item \textit{annotation}
                    \item \textit{verification and schematic creation}
                    \item \textit{schematic analysis} This is the most difficult
                    part, since the digital logic has been re-organized in
                    hierarchy forms. It can be re-created through bottom-up
                    form. Thus, a complete process of reverse-engineering can be
                    done.
                \end{itemize} 
            \end{itemize}    

\noindent \textbf{Mar 22, 2015}
\textit{2.4Gbps, 7mW All-Digital PVT-Variation Tolerant True Random Number
Generator for 45nm CMOS High-Performance Microprocessors}
\indent     \begin{itemize}
            \item cross-coupled inverter pair
            \item some experiment results need to be replicated.
            \end{itemize}

\noindent \textbf{Mar 22, 2015}
\textit{OTP}
\indent     \begin{itemize}
            \item SRAM
            \item Fuse \textbf{possible solution}
            \item Antifuse \textbf{possible solution}
            \item PROM
            \item EPROM
            \item EEPROM
            \item Flash
            \end{itemize}

\noindent \textbf{Mar 22, 2015}
\textit{Lightweight Anti-Counterfeiting Solution for Low-End Commodity Hardware
Using Inherent PUFs}
\indent     \begin{itemize}
            \item read about the way the argue about how to get rid of the doubt
            about reverse engineering
            \end{itemize}

\end{document}

\noindent \textbf{Mon 23 Mar 2015 01:15:43 PM EDT}
\textit{problems with power analysis}
\indent		\begin{itemize}
            \item Make sure the difference between power analysis and statistic
            analysis
            \end{itemize}
