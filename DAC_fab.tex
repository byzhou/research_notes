\documentclass[]{article}
\usepackage{graphicx}

\parindent=0pt
\usepackage[margin=0.5in]{geometry}

\begin{document}
\pagestyle{empty}
{\large\textbf{Research Notes}}
\begin{itemize}
    \item[*] Created on Mar 9, 2015
    \item[*] Modified on \today
    \item[*] Author info: Boyou Zhou\\
             8 St Mary's St, PHO 340, Boston, MA 02215\\
             Email: bobzhou@bu.edu, Phone: 617-678-8480
\end{itemize}


\rule[-0.1cm]{7.5in}{0.01cm}\\
\\
\noindent \textbf{Mar 9, 2015}
\textit{Verilog to gds}
\indent		\begin{itemize}
            \item \textit{Basic Steps} Synthesis, Floor-planning, and
            Place-and-Route
            \item \textit{Input} Liberty Files, timing properties of single
            gates. Lef Files, dimensions of the gates and the positions of the
            pins (Detailed strutures are irrelevant for the output of the
            design) Verilog Files or VHDL Files, the design.
            \item \textit{Output} Binary Files can be readable by
            place-and-route tool. We use Cadence Encounter. There should be a
            command to output a gds file, which contains positions of the gates
            and metal connections but no inner structures. It is possible to
            dump the gds file with the information from layer mapping files.
            Layer mapping files are defined by the technology.
        \end{itemize}

\textit{gds into virtuoso}
\indent		\begin{itemize}
            \item \textit{Input} A library, containing the designes of single
            gates. gds from par tool, including the informations of the
            positions of the gates. Name of the top level design. Technology
            files, may or may not be needed, since the library may have the
            technology files. Mapping files, may or may not be needed, since the
            technology files may include it.
            \item \textit{Output} The entire design in gds.
        \end{itemize}

\textit{My plan}
\indent     \begin{itemize}
            \item {PAR} After place-and-route our design, we dump everything with
            40nm layer mapping info.
            \item {Our Library} Create a library of our own attached with 40nm
            technology. Copy the Cellviews from Nangate. Use DRC to verify.
            \item {Import design} Import our design with our designed library.
            Worst case scenario, we need to attach our own mapping file,
            converted from technology files.
        \end{itemize}

\textit{My worries}
\indent     \begin{itemize}
            \item {Layer Mapping File} EPFL won't allow exporting Layer Mapping
            Files. Or they do not have layer mapping file. You have to use
            verilog to synthesize in EPFL. 
            \item {Dump failed} The entire design can not dumped with another
            technoloy's layer mapping file. Again, you have to resynthesis.
            \item {Copy failed} You have to draw everything yourself.
            \item {Import failed} You can first try to write your own version of
            mapping file. If that does not work, start from synthesis.
            \end{itemize}
\end{document}

