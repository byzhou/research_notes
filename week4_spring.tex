\documentclass[]{article}
\usepackage{graphicx}

\parindent=0pt
\usepackage[margin=0.5in]{geometry}

\begin{document}
\pagestyle{empty}
{\large\textbf{Research Notes}}
\begin{itemize}
    \item[*] Created on Feb 16, 2014
    \item[*] Modified on \today
    \item[*] Author info: Boyou Zhou\\
             8 St Mary's St, PHO 340, Boston, MA 02215\\
             Email: bobzhou@bu.edu, Phone: 617-678-8480
\end{itemize}


\rule[-0.1cm]{7.5in}{0.01cm}\\
\\
\noindent \textbf{Feb 16,2014}
Something more on \textit{FPGA IP protection by binding FSM to PUF}
\indent		\begin{itemize}
            \item \textit{main motivation} The idea is to get rid of the TTY (
            Trusted Third Party), or block cipher for key management and
            exchange. Previous work have been done by [3] [4] [5] [6].
            Conclusion on previous works of FPGA IP protection.
                \begin{itemize}
                \item \textit{encryption} Commercially available
                    encryption-based techniques are limited to single large FPGA
                    configuration. \textit{strange argument}
                \item \textit{encryption-based licensing} Requires TTY.
                \item \textit{HW-IP} binding methods use mechanisms in scured
                    ROM or flash memory. They are vulnerable to side-channel
                    attacks.
                \end{itemize}
            \item \textit{main idea} In the graph, it consists of core vendor,
                FPGA vendor and system developer to generate the keys. 
            \item \textit{critics} This requires the complete trust of FPGA
                vendor not involved in IP stealing.
            \item [\textbf{3}] Design security in Stratix III devices Altera White Paper
                0101, Sep. 2009
            \item [\textbf{4}] Dynamic intellectual property protection for
                reconfigurable devices, in Proc. Int. Conf. Field Programmable
                Technology, Kitakyushu, Japan, Dec. 2007
            \item [\textbf{5}] A pay-per-use licensing scheme for hardware IP cores in
                recent SRAM-FPGAs, in IEEE Trans. Information Forensics and
                Security, vol. 7, no. 1, pp. 98-108, Feb. 2012
            \item [\textbf{6}] Remote activation of ICs for piracy prevention and digital
            right management" in Proc. IEEE/ACM Int. Conf. Computer-Aided
            Design, San Jose, Nov. 2007, pp. 674-677.
        \end{itemize}

\noindent \textbf{Feb 16,2014}
\textit{Dynamic Intellectual Property Protection for Reconfigurable Devices}
\indent		\begin{itemize}
            \item \textit{main motivation} Most of the encryption of the FPGA IP
            cores are protected by static keys. The idea is to build an
            encryption system that are not based on public keys, but can only be
            licensed by private keys.
            \item \textit{main idea} So the author incorporates the
            reconfigurable ability of the FPGA to make the keys that are based
            on MAC.
            \item \textit{critics} If the SI can reverse engineer the IP, all
            the locking part can not work.
            \end{itemize}

\noindent \textbf{Feb 17,2014}
\textit{A Pay-per-Use Licensing Scheme for Hardware IP Cores in Recent
        SRAM-Based FPGAs}
\indent		\begin{itemize}
            \item \textit{reverse engineering bit streams} From the bitstream to
            the netlist, in Proc. ACM/SIGDA Symp. Field-Programmable Gate Arrays
            (FPGA), Nov. 2008, pp. 264-264
            \item \textit{taxonomy for hardware security}
                \begin{itemize}
                \item \textbf{Low Class Adversary} extract info from a plaintext
                bit stream.
                \item \textbf{Middle Class Adversary}
                A system engineer, who can get access to the purchased IP block,
                probe the internals of the IP block.
                \item \textbf{High Class Adversary}
                Pysically access to the FPGA for getting access to on-chip
                memory.
                \end{itemize}
            \item \textit{main motivation} Most of the encryption of the FPGA
            IP, the author want to propose the idea of IP protection involving
            TTP.
            \item \textit{ASIC IP protection} [19-23]
                \begin{itemize}
                \item [19] Remote activation of ICs for piracy prevention and
                digital right management, in ICCAD 2007
                \item [20] Active hardware metering for intellectual property
                protection and security in Proc. USENIX Security Symp., 2007,
                pp. 291-306
                \item [21] EPIC: Ending piracy of integrated circuits, DATE 2008
                \item [22] Protecting bus-based hard-ware IP by secret sharing,
                DAC 2008, pp. 846-851
                \item [23] Aanlysis and design of active IC metering schemes,
                IEEE workshop on hardware-oriented security and trust (HOST),
                2009, pp. 74-81
                \end{itemize}
        \end{itemize}
\end{document}

