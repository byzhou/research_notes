\documentclass[]{article}
\usepackage{graphicx}

\parindent=0pt
\usepackage[margin=0.5in]{geometry}

\begin{document}
\pagestyle{empty}
{\large\textbf{Research Notes}}
\begin{itemize}
    \item[*] Created on Mar 22, 2015
    \item[*] Modified on \today
    \item[*] Author info: Boyou Zhou\\
             8 St Mary's St, PHO 340, Boston, MA 02215\\
             Email: bobzhou@bu.edu, Phone: 617-678-8480
\end{itemize}


\rule[-0.1cm]{7.5in}{0.01cm}\\
\\

\noindent \textbf{Mar 22, 2015}
\textit{questions for ICCAD paper}
\indent		\begin{itemize}
                \item [*] Is there any papers have Hardware security problems
                published in ICCAD? How much percentages are the hardware
                security components compared to the CAD parts?
                \item Yes, there are papers. They have significant portion
                of hardware security parts.
                \item [*] What is the cost-effective solution?
                \item Memristor may not be the idea solution. It should be low
                cost thermal sensor accompanied by one time programming methods.
                The problem should be comparing the existing
                one-time-programming methods to new methods.
                \item [*] What is the state-of-art reverse engineering methods?
                \item [*] Is it possible to reverse engneering the entire
                netlist?
                
        \end{itemize}
\noindent \textbf{Mar 22, 2015}
\textit{Passive switching of eletromagnetic devices with memristors}
\indent		\begin{itemize}
        \item \textit{property} 
            \begin{itemize}
                \item The memristor directly links the charge and the flux. Thus it can
                converts flux information into charge information.             
                \item The $\beta$ constant is related to length of the memristor squared
                divided by the average ion mobility.
                \item The $\beta$ constant varies linearly with charge between the
                dopant barrier position and is bonded by the physical dimensions.
                \begin{equation}
                    R[x(t)] = x(t)R_{on}+[1-x(t)]R_{off}
                \end{equation}
                \item $R[x(t)]\in[0,1]$ defines the change from high res to low
                res.
            \end{itemize}
        \item The author is designing a rf signal applied to write the
        memristor.
        \item The rf signal applied to the memristor is pretty strong. It can
        not be written easily.
        \item It is hoped to be used as frequency selective surface.

        \end{itemize}


\noindent \textbf{Mar 22, 2015}
\textit{Temperature Tracking:An Innovative Run-Time Approach for Hardware Trojan
    Detection}
\indent		\begin{itemize}
                \item Try to decide how much percentages do the CAD components take
                compared to HT parts.
            \end{itemize}
\end{document}

