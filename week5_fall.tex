\documentclass[]{article}
\usepackage{graphicx}

\parindent=0pt
\usepackage[margin=0.5in]{geometry}

\begin{document}
\pagestyle{empty}
{\large\textbf{Research Notes}}
\begin{itemize}
    \item[*] Created on Sept 29, 2014
    \item[*] Modified on \today
    \item[*] Author info: Boyou Zhou\\
             8 St Mary's St, PHO 340, Boston, MA 02215\\
             Email: bobzhou@bu.edu, Phone: 617-678-8480
\end{itemize}


\rule[-0.1cm]{7.5in}{0.01cm}\\
\\
\noindent \textbf{Sept 29,2014}
\textit{proceedings on hardware security}
\indent		\begin{itemize}

            \item \textit{State-of-art defenses} 
                \begin{itemize}

                    \item \textit{invasive} precies measurement throughout all the gates, in order to check
                    how the gates are arranged. It is too expensive for small companies.

                    \item \textit{non-invasive} External parametric and functional IC testing. Such corresponding
                    metrics are delay, quiescent leakage, and dynamic leakage.
                    Example testing includes transient power analysis, path-delay meausurements, gate level
                    characterization, thermal profiling, or combination.

                \end{itemize}
        \end{itemize}

\noindent \textbf{Sept 30,2014}
\textit{Low Power Memristor-based ReRAM Design with Error Correcting Code }
\indent		\begin{itemize}
            \item \textit{before everything} I feel very confused is that people do not care about ecc in the 
            digital calculation circuits. I tried to find out how people have done similar work, but the result
            is none.
            \item \textit{main motivation} There are traditional error correcting code for the conventional design
            of DRAM. The author has proposed the idea to use the ECC for the memristor.
            The main idea is to use the ecc for the memristor-based memory. The author check how many bits for ecc
            is the most energy efficiency.

            \item \textit{critics} Not very innovative, it's just move all the schemes in the traditional design 
            into the memristor scheme.
        \end{itemize}

\noindent \textbf{Oct 1,2014}
\textit{Error correcting code analysis for cache memory high reliability and performance}
\indent		\begin{itemize}
            \item \textit{main concern} ECC cause a lot of area overhead and that may brings a problem.
            \item \textit{main contribution} The author proposed a tool called \textit{ad hoc} for the analysis of 
            length of the codeword and code-segment size. They propose that the idea of using Hsiao SEC-DED code is
            a good option for coding method.
            \item \textit{critics} It is a common way of ecc.
            \end{itemize}

\noindent \textbf{Oct 3,2014}
\textit{cryptography basics} \\
\indent	
            	
            \begin{itemize}
                \item Some basic terms:
                \begin{itemize}
                    \item \textit{plaintext} The information that the transmitter wants the receiver to understand.
                    \item \textit{cyphertext} The ciphered version of plaintext.
                \end{itemize}
                
                \item Classification of the encrypotography:
                \begin{itemize}
                    \item \textit{symetric encryptography} Alice and Bob has the same key. They use the same key to decrypts
                    the cyphertext.
                    \item \textit{asymetric encryptography} It contains two different kinds of keys. One is the private key
                    and the other is the public.
                        \begin{itemize}
                            \item \textit{public keys} It is based on algorithms that have no effient solutions, integer fac
                            torization, discrete logarithm and elliptic curve relations. Public keys should be able to be
                            published without compromising security.
                            \item \textit{private keys} It should not be revealed to anyone in order to maintain security. 
                            Anyone with a public can encrypt information, but only the private key holder can decrypt the 
                            information. In the Diffie-Hellman key exchange scheme, Alice and Bob can decrypt independently 
                            the cyphertext with both their own private key and the other one's public key.
                        \end{itemize}
                \item AES cryptography system\\

                    It is based on a substitution-permutation network. It is both fast in software and hardware. It operates as
                    cycles of repetition of 128, 192, 256 bit blocks. It utilize a key expansion method, Rijndael's key schedule.
                    AES requires a seperate 128-bit round key block for each round plus one more. The high level of the algorithm
                    can be summarized as the repetition of subbytes, shiftRows, mixColumns and addroundkey.

                    \begin{itemize}
                        \item \textit{subbytes} Substitute the text with another combination of text.
                        \textit{substitution}
                        \item \textit{shiftrows} For each line, shift them for one column left.
                        \item \textit{mixcolumns} Use exclusive-or to mix the plaintext with the pseudo-random codes.
                        \textit{diffusion}
                    \end{itemize}
                    
                \end{itemize}

            \end{itemize}
\end{document}

