\documentclass[]{article}
\usepackage{graphicx}

\parindent=0pt
\usepackage[margin=0.5in]{geometry}

\begin{document}
\pagestyle{empty}
{\large\textbf{Research Notes}}
\begin{itemize}
    \item[*] Created on Feb 8, 2015
    \item[*] Modified on \today
    \item[*] Author info: Boyou Zhou\\
             8 St Mary's St, PHO 340, Boston, MA 02215\\
             Email: bobzhou@bu.edu, Phone: 617-678-8480
\end{itemize}


\rule[-0.1cm]{7.5in}{0.01cm}\\
\\
\noindent \textbf{Feb 8,2014}
\textit{Invasive PUF Analysis}
\indent		\begin{itemize}
            \item \textit{main motivation}
            The paper is similar to the work that we have done. They are using
            the laser as the stimulation and try to detect the HT by testing the
            response. The difference is that they are focusing on the material
            level and we are working on the circuit level. They also try to find
            a way to have photonic stimulation but with electric response, which
            I think that it is a cool idea that can be explored in the future.

            \item \textit{main idea}
            The main work has been done in the area of SRAM PUF. The technique
            has been divided into two part. One part is the
            optical-stimulation-electric-response and the other is to use
            optical technique to detect the stored info inside the SRAM.
            
            The most awesome part of this paper is the idea of OSER. Once the
            laser has been applied on the gate, the gate can be turned on. With
            this technique, the idea can be used in gate-by-gate testing.

            \item \textit{critics}
            The author has met some problems to apply the idea. The main issue
            is that what we are dealing with is digital circuit. The shot
            current will lead to noise ground that make it hard to detect the
            electrical response. My first order idea to solve this idea is to
            encode the stimulation. The electrical response must be processed in
            digital region.
        \end{itemize}

\noindent \textbf{Feb 12,2014}
\textit{Optical PUFs reloaded}
\indent		\begin{itemize}
        \item \textit{main motivation}
            Low cost PUFs are designed compared to electrical PUFs. They are
            randomly distributed light scatterers inside of circuits. The
            authors analyzed the non-integrated PUF constructed from inexpensive
            componenets. This paper talks a lot about optical PUFs. So I guess
            all the citations should be reviewed.
          
        \item \textit{main idea}
            Optical PUFs provide massive amount of infos for the test, which
            electrical PUFs do not have. This make the ML much harder to predict
            the behaviors. The author also focus on the reflected information
            processing. The author highlight the idea of increase the response
            entrophy.

        \item \textit{critics}
            The author only focus on the image processing process instead of the
            pattern design. The overhead of the integrated PUFs is too much.
            
        \end{itemize}

\noindent \textbf{Feb 13,2014}
\textit{FPGA IP protection by binding finite state machine to physical
unclonable function}
\indent     \begin{itemize}
        \item \textit{main motivation} 
            There are basicly three main ways of preventing IP cloning. The
            first way is to use watermarking. But watermarking is a passive way
            to solve the problem. It can not prevent people from cloning
            actively. The second way is to lock the IP, which requires keys to
            unlock it. The thing is that with side channel attacking, it is
            possible to retrieve the keys and then cloning without notice. The
            last method is using the PUF. PUF has an advantage of device
            dependent. For different devices, the key is different.
        \item \textit{main idea} Talks about how to use PUF to prevent IP
            cloning. 
        \item \textit{critics} The paper has not addressed the issue with
            complete reverse-engineering of the entire chip. My point of view is
            that if it is possible to use the bit-stream to reverse engineering
            the entire chip and then it is impossible to prevent IP
            counterfeiting.
        \end{itemize}
\end{document}



