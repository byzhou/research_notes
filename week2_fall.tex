\documentclass[]{article}
\usepackage{graphicx}

\parindent=0pt
\usepackage[margin=0.5in]{geometry}

\begin{document}
\pagestyle{empty}
{\large\textbf{Research Notes}}
\begin{itemize}
    \item[*] Created on Sept 6, 2014
    \item[*] Modified on \today
    \item[*] Author info: Boyou Zhou\\
             8 St Mary's St, PHO 340, Boston, MA 02215\\
             Email: bobzhou@bu.edu, Phone: 617-678-8480
\end{itemize}


\rule[-0.1cm]{7.5in}{0.01cm}\\
\\
\noindent \textbf{Sept 6,2014}
\textit{Wide I/O with 4096b TSVs Through an active silicon interposer with in-place waveform capturing}
\indent		\begin{itemize}
            \item \textit{main motivation} The reason why 3D stacking is not applicable towards the massive production is that TSV is to expensive for massive production.
            The interposer is active, it can be controlled actively through the control. 
            The main point is to use the interposer to detect the waveform throught the TSV in order to build up a test agent to test the communication between different layers.
            \item \textit{main idea} The idea is that the chip needs massive amount of veritical connections.
            The silicon interposers in the stack provide pitch in horizontal and veritical channels for the accomodating TSVs and uBunmps.
            In the paper, the author states well about how the chip is going to test and describe the testing agent function's behavior.
            \item \textit{critcs} The BIST is through the traditional design and most of the ideas are to modify the current idea into an interposer.
            Personally, I think it is a good paper to present the idea to work in the future, but the application for the 3D is very limited.
        \end{itemize}

\noindent \textbf{Sept 7,2014}
\textit{An 8MHz 75uA/MHz Zero leakage non-volatile logic based cortex mo MCU Soc exhibiting 100percent digital state retention at vdd=0}
\indent		\begin{itemize}
            \item \textit{main motivation} 
                TI proposed an idea to use the logic based memory which is also non-volatile.
                The system also reduces the wake up time when the vdd starts to power up from 0 voltage.
               
            \item \textit{main idea} 
                The idea is to use capacitor to store the information.
                The best part of the paper is that it use the control signal to control the writing operation towards the NV cells.
                The effective driver will accelerate the read and write time.
                
            \item \textit{Critics}
                The price pay to the area is enormouse. 
                The entire on cell has been increased into several capacitors.
                I did not lookinto this paper, but I think it is a good paper to start to understand fast read and write to NV cells. 
        \end{itemize}

\noindent \textbf{Sept 8,2014}
\textit{A High Performance eleiptic curve cryptographic processor over GF(p) with SPA Resistance}
\indent		\begin{itemize}

            \item \textit{SPA} Simple Power Analysis. It is related to the security problem of the hardware.\textbf{SPA attacks}

            \item \textit{Montgomery modular multiplication} It is an algorithm that proves to be much more efficient after several hundreds of bits.\textbf{interesting algorithm}
            
            \item \textit{eleiptic curve crytographic system} It is another cryptographic systems as an alternative towards RSA system. It can achieve 160 bits security as RSA can do
                in 1024 bit. Personally, I support to use the state of art cryptology for the hardware securty testbench. Once again, the cryptology involves the finite field study.
                Thus, I think, it is very important to read through the finite field study.

            \item \textit{main motivation} SPA has presented a great threat for hardware devices of public cryptosystems. By analyzing the power consumption, it is easy to detect
                the private key's value. That is why the paper proposes a better way to counter SPA. The Montgomery modular multiplication is used for high speed modular computation, which
                is needed in encryptography.

            \item \textit{Basic idea} The paper is about use the ECCS to build a system that can counter SPA. The idea is to build it.

            \item \textit{Critics} I do not know much about the security part. But I am sure there should be a way to breach it. Anyway, it is good way to start looking into the security
                problems. Understanding the public keys and private keys.

            \end{itemize}

\noindent \textbf{Sept 9,2014}
\textit{A 65 nm 39GOPS/W 24-Core Processor with 11Tb/s/W Packet-Controlled Circuit-Switched Double-layer Network-on-chip and heterogeneous execution array}
\indent		\begin{itemize}
            \item \textit{Main motivation} The processor has been organized in 2D mesh.
            Each cluster contains its own VCO, which confused me that how the clock can be synchronized.
            The system contains aggressive clock gating. The cores are the same, but one has reconfigurable arrays.

            \item \textit{Basic idea} The idea of the paper is to use the packet switching to control circuit switching.
            Circuit switching is for a higher energy efficiency.
            The packet switching is extremely efficient in the small amount of data. \textbf{doubt}
            The circuit switching has a higher efficiency in energy saving.\\

            The processor has used x-y dimension ordered routing algorithm.\textbf{place and route algorithm}
            The clock signal has been transmitted with the data bits so that synchronization has been achieved.
            The data and clock will be resynchronized after several nodes.
            These guys have set up a reconfigurable parts for flexible connections and configurable units.
            These units basic for re-routing to some special functional units for some complex computation.

            \item \textit{critics} Awsome idea on the core-to-core communication for the circuit level.
            The processor did not measure IPC for the efficiency of the execution.
            I guess it is not good, because the architecture is pretty simple.
            It is using RISC for the main core.

            \end{itemize}

\noindent \textbf{class notes, Sept 9, 2014}
\textit{PCB project}
\indent		\begin{itemize}

            \item   file $>$ new $>$ project $>$ PCB project
            \item   file $>$ new $>$ libary $>$ schematic 
            \item   Goes to digikey and find the component that you need. Pick up the library component first.
                    In the end of the datasheet, we can see the schematic.
            \item   Tools $>$  IPC Footprint Wizard
            \item   package overall dimension. Sometimes it needs to specify the max and min dimension. SOT23 on the first page on the datasheet.
                    Pick the SOT23 so that you can sold it easily.
            \item   Wizard will goes through some parameter that should be default. 
            \item   Footprint Dimensions. It will have pad dimension that we can make it a little comfortable to do it.
            \item   Put it in the library.

            \item   Purple means the exposed metal layer. Red means hidden metal layers. Blue means middle. Overlay layer means notes.
                    Top solder and bottom solder for sold.
            \item   Two layers of PCB.

            \item   Create a non-std footprint.
            \item   PCB library. It can show the form.
            \item   Add another awesome item.
            \item   Use the button q. It can change from inch to mm or mm to inch.
            \item   The pad can goes through and goes to a bottom.
            \item   Double click the pad. The hole should be 130 100.
            \item   In the Layer option, you can change the pads through the board. The via does not appear in the library.
            \item   x for fliping, space for rotating.
            \item   Library Component Properties.Default Designator is the name of the components. You are going to write U?. Model with schematics.
            \item   In the schematics, place port. Use the tab key, it will come up with the proporties of pins. Use the \ for the name of the
                    port.
            \item   Source document. Make a new schametics. Place parts that we created in the library.
            \item   Place wires. The cross will become a large x to indicate the connection.
            \item   Place labels. Put the value of the components.
            \item   PgUp and PgDown are zoom in or zoome
            \item   Tools $>$ take schametics. It needs individual names. Annotate schametics.
            \item   Right click on something and Update schematic and the footprint name will be consistant with the updated library.
            \item   There is a global ground to connect all the stuff.
            \item   File $>$ new PCB
            \item   Grid can be changed.
            \item   Make two boards.
            \item   In mechanical 1, I can draw the boundaries.
            \item   Design $>$ Boundary 
            \item   Design $>$ Import blabla design
            \item   Top level to down. 
            \item   Routing part from the top to down. Interactive connecting.
            \item   Design $>$ layer stack manager.
            \item   Tools $>$ Design Rools Check
            \item   Garber setup. Send this to fab.
            \item   NC Drill file. This is for holes. Use the binary file.
            \item   Bill material.

            \end{itemize}

\noindent \textbf{ Sept 10, 2014}
\textit{Energy-Efficient Clocking Based on Resonant Switching for Low-Power Computation}
\indent		\begin{itemize}

            \item \textit{Landauer's principle concerning irreversible logic} There is a minimum energy dissipation of $k_{B}Tlog(2)$ when information is erased.
            \item \textit{EERS} Energy recovery resonant switching.    
        
            \item \textit{main motivation} Reducing the energy consumption by using resonant for clocking signal.

            \item \textit{main idea} Generate a clock signal with very low cost of energy by implementing the resonant circuit. 

            \item \textit {critics} The frequency of the circuit is going to be low.
            The use of inductor in the IC is very costly. I believe the author did not tape out the chip.

            \end{itemize}

\noindent \textbf{ Sept 11, 2014}
\textit{A 200mV 32b Subthreshold Processor with Adaptive Supply Voltage Control}
\indent		\begin{itemize}
            \item \textit{main motivation} The circuit needs to work under a much lower supply voltage in order to save energy. 
            The author proposed that current ways of voltage or frequency is overly pessimistic.
            It needs a PFD ( phase / frequency) detector for detecting the errors.




            \end{itemize}
\end{document}

