\documentclass[]{article}
\usepackage{graphicx}

\parindent=0pt
\usepackage[margin=0.5in]{geometry}

\begin{document}
\pagestyle{empty}
{\large\textbf{Research Notes}}
\begin{itemize}
    \item[*] Created on Oct 27, 2014
    \item[*] Modified on \today
    \item[*] Author info: Boyou Zhou\\
             8 St Mary's St, PHO 340, Boston, MA 02215\\
             Email: bobzhou@bu.edu, Phone: 617-678-8480
\end{itemize}


\rule[-0.1cm]{7.5in}{0.01cm}\\
\\
\noindent \textbf{Oct 27,2014}
\textit{High-level Synthesis for Run-time Hardware Trojan Detection and
Recovery}
\indent		\begin{itemize}
            \item \textit{main motivation} The author does not like the idea
            with side channel analysis. The author propose the idea to detect
            the trojan and then recover the chip to the normal state. 
            \item \textit{IP core trojan insertion} Two funciton-equivalent
            using IP cores from different vendors carrying the same trojan is
            not very likely.  
            \item \textit{NC RC} Normal Computation and Re-Computation
            \item \textit{related work} NC and PC computation's inconsistancy
            will be viewed as the insertion of harware trojan.
            \item \textit{architecture asurance} It's like the Boeing747 using
            three architectures to ensure the accuracy of computation.
            \item \textit{ILP} The auther considers the cost problem as an
            optimization problem. Pretty awesome they convert binary problem to
            a linear optimization problem.
            \item \textit{critics} Cost paid for getting two identical IP
            cores.
            \item \textit{high-level synthesis} GAUT, a high level synthesis
            tool.
        \end{itemize}
\end{document}

