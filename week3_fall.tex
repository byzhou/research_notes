\documentclass[]{article}
\usepackage{graphicx}
%coffee stain package
\usepackage{coffee3}
\usepackage{tikz}
\usepackage{verbatim}
\usetikzlibrary{arrows,shapes}

\parindent=0pt
\usepackage[margin=0.5in]{geometry}

\begin{document}
\pagestyle{empty}
{\large\textbf{Research Notes}}
\begin{itemize}
    \item[*] Created on Sept 15, 2014
    \item[*] Modified on \today
    \item[*] Author info: Boyou Zhou\\
             8 St Mary's St, PHO 340, Boston, MA 02215\\
             Email: bobzhou@bu.edu, Phone: 617-678-8480
\end{itemize}

%\cofeA{0.75}{0.5}{90}
%This is used for adding a coffee stain.
\rule[-0.1cm]{7.5in}{0.01cm}\\
\\
\noindent \textbf{Sept 15,2014}
\textit{A 4.5Tb/s 3.5Tb/s/W 64*64 Switch Fabric With Self-Updating Least-Recently-Granted Priority and Quality-of-service
        Arbitration in 45nm CMOS}
\indent		\begin{itemize}

            \item \textit{main motivation} For high speed and low energy consumption purpose, the conventional routers
            can not satisfying the request for large building blocks. Conventional block uses distinc blocks to build
            up the topology. This becomes a bottleneck.
            \item \textit{main idea} The proposed SSN has multicast using least recently granted priority. They claim 
            that the LRG and Qos has zero coding overhead. The idea is to design a priority line in the system for
            the communication to use the priority info to determine the driver and receiver. The one line with the 
            highest priority wins the line and then lose all priority.
            \item \textit{critics} The idea for evenly distributes the priority is to use the Most Recent used gets the
            the highest priority. In my opinion, there can be better evenly distributed priority algorithm to provide a
            better performance. But in this case, low energy consumption is the priority.

        \end{itemize}

\noindent \textbf{Sept 16,2014}
\textit{Centip3De: A Cluster-Based NTC Architecture With 64 ARM Cortex-M3 Cores in 3D Stacked 130nm CMOS }
\indent     \begin{itemize}

            \item \textit{motivation} Demo the idea of combination of 3D stacking and near-threshold stacking on a larger
            scale. In the NTC region, it is very hard to provide a higher performance.

            \item \textit{cholesky decomposition} It is a factorization for solving systems of linear equations.

            \item \textit{main idea} The best idea is to use four out-of-phase 90 degrees clock signals for four cores so
            that it can be viewed as one single cycle from outside. They share the same cache.
            The cache clock is much faster than the cores. That is a very interesting idea.

            \item \textit{critics} This idea can only expands to 4 core. I don't believe it can be expands to larger scale.
        \end{itemize}

\noindent \textbf{Sept 17,2014}
\textit{A 2.05GVertices 151mW Lighting Accelerator for 3D Graphics Vertex and Pixel Shading in 32nm CMOS}
\indent     \begin{itemize}
            \item \textit{main motivation} Reduce energy and have higher performance. The author proposes that hardware
            accelerator is important for performance improvement.
            \item \textit{main idea} The author proposes to complete the calculation on the log domain. And then use FPWL
            (fixed point piecewise linear approximation log circuit). The logrithm is approximated by 5 linear intervals.
            The intervals approximation introduces errors to the system but this is for graphics application, so the 
            efficiency of power is priority.
            \item \textit{critics} It is a great idea for the graphics applications. If the error control can be less, it
            may be applied to the GPP.
        \end{itemize}
\end{document}

